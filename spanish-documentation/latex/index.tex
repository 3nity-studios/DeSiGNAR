Este es el comienzo de un proyecto completo el cual se espera sea una biblioteca general para ser usada en el desarrollo de modelos y como herramienta para simulaciones.

La estructura de esta librería es, por directorios\+:

\tabulinesep=1mm
\begin{longtabu} spread 0pt [c]{*{2}{|X[-1]}|}
\hline
\rowcolor{\tableheadbgcolor}\PBS\centering \textbf{ Directorio }&\textbf{ Descripción  }\\\cline{1-2}
\endfirsthead
\hline
\endfoot
\hline
\rowcolor{\tableheadbgcolor}\PBS\centering \textbf{ Directorio }&\textbf{ Descripción  }\\\cline{1-2}
\endhead
\PBS\centering {\itshape include} &Contiene todos los archivos de cabecera. \\\cline{1-2}
\PBS\centering {\itshape src} &Contiene todos los archivos fuente (implementaciones declaradas en las cabeceras) \\\cline{1-2}
\PBS\centering {\itshape samples} &Contiene algunos test y demos con instrucciones de uso de las distintas abstracciones implementadas. \\\cline{1-2}
\PBS\centering {\itshape obj} &En este directorio serán creados todos los archivos objeto cuando sea compilada la biblioteca. \\\cline{1-2}
\PBS\centering {\itshape lib} &Cuando se compila la biblioteca, en este directorio será creado el archivo {\bfseries lib\+Designar.\+a}. \\\cline{1-2}
\PBS\centering {\itshape bin} &Cuando sea compilada la biblioteca, en este directorio serán creados los archivos binarios a ejecutar. \\\cline{1-2}
\end{longtabu}
Para compilar la biblioteca, se debe ejecutar el comando\+:


\begin{DoxyCode}
~$ make library
\end{DoxyCode}


Para compilar los samples, se debe ejecutar el comando\+:


\begin{DoxyCode}
~$ make samples
\end{DoxyCode}
 